
\documentclass[sigconf]{acmart}


\settopmatter{printacmref=true}

\usepackage{algorithm}
\usepackage{algpseudocode}
\usepackage{balance}
\usepackage{listings}
\usepackage{setspace}
\usepackage{xspace}

\RequirePackage{listings}


\lstset{
  language=C++,
  numbers=left,
  stepnumber=1,
  numbersep=0.5em,
  basicstyle=\ttfamily,
  keywordstyle=\color{blue}\textbf,
  stringstyle=\color{red}\ttfamily,
  commentstyle=\color{green}\ttfamily,
  morecomment=[l][\color{magenta}]{\#}
  frame=single,
  breaklines=true,
}


\def\BibTeX{{\rm B\kern-.05em{\sc i\kern-.025em b}\kern-.08emT\kern-.1667em\lower.7ex\hbox{E}\kern-.125emX}}
 

\copyrightyear{2019}
\acmYear{2019}
\acmConference[FEAST'19]{3rd Workshop on Forming an Ecosystem Around Software
Transformation}{November 15, 2019}{London, United Kingdom}
\acmBooktitle{3rd Workshop on Forming an Ecosystem Around Software Transformation
(FEAST'19), November 15, 2019, London, United Kingdom}
\acmPrice{15.00}
\acmDOI{10.1145/3338502.3359760}
\acmISBN{978-1-4503-6834-6/19/11}

%\copyrightyear{2019}
%\acmYear{2019}
%\setcopyright{acmcopyright}
   % adjust this to the correct option per the rightsreview. Provided in ACM rightsreview confirmation email.
%\acmConference[CCS '19] {2019 ACM SIGSAC Conference on Computer and Communications Security}{November 11--15, 2019}{London, United Kingdom}
%\acmBooktitle{2019 ACM SIGSAC Conference on Computer and Communications Security (CCS'19), November 11--15, 2019, London, United Kingdom}
%\acmPrice{15.00}
%\acmDOI{10.1145/XXXXXX.XXXXXX}
%\acmISBN{978-1-4503-6747-9/19/11} 


\input{macros.tex}

\begin{document}

\fancyhead{}

\title{Automated Finite State Machine Extraction}



\author{Yongheng Chen}
\authornote{The work was done when Yongheng Chen was a visiting student at Pennsylvania State University.}
\email{changochen1@gmail.com}
\affiliation{%
  \institution{NJU, Pennsylvania State University}
}

\author{Linhai Song}
\email{songlh@ist.psu.edu}
\affiliation{%
  \institution{Pennsylvania State University}
}

\author{Xinyu Xing}
\email{xxing@ist.psu.edu}
\affiliation{%
  \institution{Pennsylvania State University}
}

\author{Fengyuan Xu}
\email{fengyuan.xu@nju.edu.cn}
\affiliation{%
 \institution{Nanjing University}
}

\author{Wenfei Wu}
\email{wenfeiwu@outlook.com}
\affiliation{%
 \institution{Tsinghua University}
}


\hyphenation{FSM-Extractor}

\def\ie{\emph{i.e.}}
\def\eg{\emph{e.g.}}

\newcommand{\hpad}[0]{\hspace*{\fill}}

\newcommand{\Tool}{FSMExtractor}

\newcommand{\Thrust}[1]{\hyperref[sec:thrust-#1]{Thrust #1}}
\newcommand{\Code}[1]{\lstinline{#1}}
%
% The abstract is a short summary of the work to be presented in the article.
\begin{abstract}
Finite state machine (FSM) is a type of computation models widely
used in various software programs.
Extracting implemented FSMs has many important applications in the
networking, software engineering and security domains.
In this paper, we first conduct an empirical study to understand 
how FSMs are implemented in real-world software.
Under the guidance of our study results, we then design a static analysis tool, 
\Tool{}, to automatically identify and synthesize implemented FSMs.
Evaluation using 160 software programs from three sources shows that
\Tool{} can extract all implemented FSMs 
and report very few false positives.




\end{abstract}

%
% The code below is generated by the tool at http://dl.acm.org/ccs.cfm.
% Please copy and paste the code instead of the example below.
%
\begin{CCSXML}
<ccs2012>
<concept>
<concept_id>10011007.10010940.10010992.10010998.10011000</concept_id>
<concept_desc>Software and its engineering~Automated static analysis</concept_desc>
<concept_significance>500</concept_significance>
</concept>
<concept>
<concept_id>10011007.10011074.10011099.10011102.10011103</concept_id>
<concept_desc>Software and its engineering~Software testing and debugging</concept_desc>
<concept_significance>500</concept_significance>
</concept>
<concept>
<concept_id>10003033.10003039.10003041.10003042</concept_id>
<concept_desc>Networks~Protocol testing and verification</concept_desc>
<concept_significance>300</concept_significance>
</concept>
</ccs2012>
\end{CCSXML}

\ccsdesc[500]{Software and its engineering~Automated static analysis}
\ccsdesc[500]{Software and its engineering~Software testing and debugging}
\ccsdesc[300]{Networks~Protocol testing and verification}

%
% Keywords. The author(s) should pick words that accurately describe the work being
% presented. Separate the keywords with commas.
\keywords{finite state machine, static analysis}


% This command processes the author and affiliation and title information and builds
% the first part of the formatted document.
\maketitle

\section{Introduction}

A finite state machine (FSM) is a mathematical model of computation, 
which performs a series of predetermined actions in 
reaction to the model inputs~\cite{fsm}. 
FSMs provide a concise and expressive way to describe
program logic, so that they widely exist in various software programs, 
including network protocols, 
compiler and event-driven programs. 

Automatically extracting implemented FSMs in a program has 
many important applications. 
First, the implementation of a FSM may be inaccurate. By 
 comparing the extracted version with its orginal design, one 
can detect potential mistakes residing in an implementation~\cite{protocol-bug}. 
Second, the network verification process depends on underlying FSM models 
in different components to validate the whole network properties 
(i.e. isolation, reachability).
Generally, the FSMs fed into verification are largely handcrafted through 
manual inspection~\cite{fayaz2016buzz,SymNet}, which is time-consuming and error-prone. 
Third, extracted FSMs can help developers and automated program analysis tools
better understand program semantics, facilitating the building of future 
code debloating~\cite{container-debloating-1,container-debloating-2,dinghao-1} 
and fuzzing techniques~\cite{afl,Angora,youwei-1}.


Unfortunately, there has not yet been an existing algorithm that can extract 
all implemented FSMs in a program.
Static techniques~\cite{wu2016automatic,khalid2016paving} 
can extract certain models implemented in a program,
but their models are less concise and expressive than FSM. 
Dynamic techniques~\cite{angluin1987learning,moon2019alembic,cho2011mace} 
view the whole programs as a blackbox and 
model it as one single FSM in a coarse granularity, 
failing to extract all FSMs and localize program code 
pertaining to an implemented FSM. 
Dynamic techniques highly depend on used inputs during FSM inference, 
lacking soundness and completeness.  

In this work, we presents a tool \Tool{} that can effectively extract FSMs in a program
with good coverage and accuracy. Technically,  
\Tool{} utilizes static analysis techniques, takes program code as input, 
and outputs a five-element tuple ($Q$, $\sum$, $\delta$, $s_0$, $F$) 
describing each identified FSM. 


We build \Tool in two steps. First, we conduct an empirical study 
on how FSMs are implemented in the real world. After examining 25 FSMs in the CGC 
dataset~\cite{CGC}, we find that there are clearly code patterns for FSM implementations. 
For example, all of our studied FSMs are implemented in 
loops which do not take a constant trip count, and a state transition operation 
is control dependent on the current state. 
Second, we design static analysis routines for the code patterns.
Our static analysis techniques can recognize suspicious FSM loops,
recognize variables representing FSM states, and synthesize the five-element tuples. 
Our evaluation using 160 programs from three sources shows that
\Tool{} can identify all implemented FSMs with very few false positives. 


%The rest of the paper is organized as follows. 
%In Section~\ref{sec:study}, 
%we discuss our empirical study on how FSMs are implemented in real-world software.
%In Section~\ref{sec:impl}, we discuss the detailed design of \Tool{}.
%How we conduct experiments to evaluate \Tool{} is discussed in Section~\ref{sec:exp}. 
%We discuss the potential applications of extracted FSMs in Section~\ref{sec:app}
%In the last two sections, we discuss related works and conclude our paper. 




\section{Empirical Study}
\label{sec:study}
In this section, we will first review the mathematical definition of FSM and then
we will describe our empirical study on how FSMs are implemented 
in real-world software. 

\noindent\textbf{Background.}
A finite state machine (FSM) is a mathematical computation model, 
which consists of several internal states and takes inputs from the outside.
At any time, a FSM can only be in one state. 
When a certain condition is satisfied, 
a FSM transits from one state to another state. 
A FSM can be specified using a five-element tuple ($Q$, $\sum$, $\delta$, $s_0$, $F$),
where $Q$ is a set of internal states, $\sum$ is an input alphabet, 
$\delta$ is a set of transition functions,
$s_0$ is the initial state, and $F$ is a set of final states. 

\noindent\textbf{Real-World Implementation.}
We leverage the DARPA CGC dataset~\cite{CGC} to 
understand how FSMs are implemented in the real world. 
We choose the CGC dataset, because it 
contains a large number of diverse programs simplified 
from real-world software and it 
is also widely used in security 
community~\cite{QSYM, Driller, VUzzer}. 


To conduct the study, we first randomly sample 
40 programs from the CGC dataset.
We then manually inspect the sampled programs and look for implementation of FSMs.
In total, we identify 25 implemented FSMs, 
and they are the targets of our study.

Figure~\ref{fig:cgc-fsm} shows one example of studied FSMs.
Function \texttt{cgc\_parse\_set()} takes string \texttt{right} 
as input and returns \texttt{true} if \texttt{right} matches 
regular expression ``\verb/|("[^"]*")?|/''. 
Figure~\ref{fig:cgc} shows the implemented FSM. 
In total the FSM contains six different states 
and nine possible state transitions. 

{
\begin{figure}[h]
\begin{minipage}{\columnwidth}
\begin{center}
\scriptsize
\lstinputlisting[xleftmargin=.15in,language=C++,basicstyle=\ttfamily,keepspaces=true]{figure/cgc_fsm.c}
%\caption{A data race caused by anonymous function.}
%\label{fig:docker27037}
\mycaption{fig:cgc-fsm}{A simplified FSM implementation from the CGC dataset.}
%{The code has been simplified for illustration purpose.}
{}
\end{center}
\end{minipage}
\end{figure}
}


\begin{figure}[htb]
\centering
\includegraphics[width=2.5in]{figure/cgc-fsm.pdf}
%\vspace{-1.5em}
\mycaption{fig:cgc}{An implemented FSM in the CGC dataset.}
%{The code has been simplified for illustration purpose.}
{}
\end{figure}

To guide the implementation of \Tool{}, our empirical study 
is mainly conducted from the following aspects.


First, what code constructs are used to implement the studied FSMs?
Since our goal is to statically identify and extract implemented FSMs, 
we must know what code constructs to inspect. 
Not surprisingly, all our studied FSMs are implemented using a loop, 
like the \texttt{while} loop at line 13 in Figure~\ref{fig:cgc-fsm}.  
In each loop iteration, an implemented FSM processes an input and 
determines whether to stay in the current state or transit to a new state. 
The underlying intuition is that a FSM usually needs to process 
multiple inputs and similar logics are applied during the processing, 
so that using loop is a natural way to implement a FSM. 

Another important observation is that 
the FSM loops do not execute constant 
iterations or take constant trip counts.
Their executions dynamically depend on inputs, 
since it is very rare that a FSM can arrive at a final 
state after processing a predefined, constant number of inputs. 
For example, the iteration number of the 
loop in Figure~\ref{fig:cgc-fsm}
is not constant and it
depends on the content of input string \texttt{right}.

Second, how internal states ($Q$) are maintained by the FSMs?
Intuitively, there must be a state variable, which tracks the current state of a FMS.
When state transition happens, 
the value of the state variable is changed. 
Our study confirms this intuition. 
We also find that state variables are either in integer type or enumeration type,
and their values are discrete and bounded in a certain range. 
This finding indicates that static value set analysis~\cite{DEEPVSA,VSA} 
can potentially determine all possible states of an implemented FSM.
For example, local variable \texttt{state} declared at line 11
is the state variable of the FSM in Figure~\ref{fig:cgc-fsm}.
It is in enumeration type.
In total, it has six possible values 
specified by its type declaration at line 1, 
corresponding to the six states in Figure~\ref{fig:cgc}.
Interestingly, one studied FSM loop contains two state variables,
and this case reminders us that developers could use one loop 
to implement multiple FSMs. 
We need to extract all of them when designing \Tool{}.



Third, what is the input alphabet ($\sum$)? 
The input alphabet of a FSM is theoretically bounded by all possible values 
of the data type used to represent inputs. 
For example, the input alphabet of the FSM 
in Figure~\ref{fig:cgc-fsm} contains all possible byte values.
There are also cases where an input alphabet is a subset of all possible values, 
and we think value set analysis can help refine 
the input alphabet for an identified FSM.

We observe that a FSM loop processes a distinct input in each iteration. 
Sometimes, a FSM loop needs to refer to a different memory location for a new input. 
Sometimes, a new input is written to the same 
location in each iteration before the FSM loop starts its procession.    
For example, \texttt{right} points to the input character 
processed by the FSM in each iteration. 
The value of \texttt{right} is incremented by one at line 38 in each iteration, 
so that the FSM reads a different memory location in each iteration. 

We also observe that during the implementation of a FSM, 
developers usually do not enumerate the processing rule for every possible
input value, 
and they tend to explicitly specify the rules only for several special values
and leave others to be handled by a default rule. 
For example, only the processing rules for `\verb/|/' and `''' 
are explicitly specified in Figure~\ref{fig:cgc-fsm}, 
and all other byte values are handled by 
the default rule at line 31. 

Four, how the transition functions ($\delta$) are implemented?
Transition functions take the current state and a value
in the alphabet as input and output the next state. 
A transition function is executed in each iteration, 
the FSM relies on the output value to decide whether to transit to a new state. 
We observe that transition functions are implemented
using control constructs (e.g., \texttt{if}, \texttt{switch}). 
For example, a transition function in Figure~\ref{fig:cgc-fsm} 
is implemented in line 14, 15, 16, and 17. 
If the current state is \texttt{start} 
at line 16 and the current input value is 
`\verb/|/' at line 14, the transition function outputs \texttt{open\_set}
as the next state at line 17. 
Line 22, 23, 24, and 25 implement another transition function,
which consumes an input character `''' at line 22 and 
transits from the current state 
\texttt{open\_double} at line 24 to 
the next state \texttt{close\_double} at line 25. 

Five, how to specify the initial state ($s_0$) and the final states ($F$)? 
$s_0$ of a FSM can be specified 
by the value of the state variable before the execution of the FSM loop.
For example,  $s_0$ of the FSM in Figure~\ref{fig:cgc-fsm}
is \texttt{start}, which is the value of \texttt{state} 
before the loop execution at line 13. 
When a FSM loop finishes its execution, 
all possible values of the state variable
represent $F$ of the FSM. 
For example, the \texttt{while} loop in Figure~\ref{fig:cgc-fsm} terminates 
its execution when finishing parsing string \texttt{right}, 
so that \texttt{state} can be any of the six values 
in the type declaration at line 1 
and any state of the FSM in 
Figure~\ref{fig:cgc} can be a final state. 


To sum up, our empirical study shows that 
there are clear code patterns used by developers to implement FSMs. 
In Section~\ref{sec:impl}, we will discuss how we 
leverage these patterns to build \Tool{}, 
which can automatically extract implemented FSMs from a program. 






\section{Implementation}
\label{sec:impl}

Our empirical study in Section~\ref{sec:study} 
shows that a FSM is implemented in a loop 
which does not have a constant trip count (or iteration number) 
and conditionally updates a state variable 
to transit to a new state in each iteration. 
Therefore, \Tool{} searches FSM loops
by first filtering loops with constant trip counts (Section~\ref{sec:constant}) 
and then identifying loops with state variable updates (Section~\ref{sec:variable}).
The ultimate goal of \Tool{} is to extract implemented FSMs, 
and thus we will discuss how \Tool{} figures out the five-element tuple 
($Q$, $\sum$, $\delta$, $s_0$, $F$) 
for an identified FMS in Section~\ref{sec:tuple}. 

Algorithm~\ref{alg:fsm} shows the workflow of \Tool{}.
\Tool{} takes the source code of a program as input
and outputs the five-element tuple (line 8) 
and source code information (line 9) for each 
implemented FSM in the program.  


\begin{algorithm}[!htb]
    \caption{Finite State Machine Extraction}
    \label{alg:fsm}
    \begin{algorithmic}[1]
        \Require source code of a program: \emph{P}
        \Function {\Tool}{$P$}
        \State initialize an empty FSM set \emph{S} = \{\}
        \For{each loop $l$ in $P$}
        	\If{$l$ has a constant trip count}
        		\State \textbf{continue}
        	\EndIf
        	 \If{$l$ has state variable updates}
        			\State t($Q$, $\sum$, $\delta$, $s_0$, $F$) $\gets$ ConstructFSM($l$)
        			\State locInfo $\gets$ ExtractLOCInfo($l$) 
        			\State S.Insert(t, locInfo)
        	\EndIf
        \EndFor
        \State \Return{$S$}
        \EndFunction
    \end{algorithmic}
\end{algorithm}


\subsection{Filtering Loops with Constant Trip Counts}
\label{sec:constant}
As discussed in Section~\ref{sec:study},
a FSM is usually implemented using a loop 
and the loop processes one input in each iteration to decide 
whether to transit to a new state. 
In reality, it is very rare that a FSM can arrive at a final state 
after processing a constant number of inputs.
Our empirical study confirms this intuition. 
None of our studied FMS loops take a constant trip count.  
To sum up, given a loop which iterates a constant 
number in each execution, 
the loop is unlikely to be a FSM implementation. 

We mainly leverage scalar evolution analysis~\cite{scalar-1,scalar-2,scalar-3} 
to identify loops whose trip counts are constant. 
Scalar evolution analysis can identify reduction variables inside a loop.
Reduction variables are integer variables, 
whose values are updated 
with a constant delta in each loop iteration. 
For example, variable \texttt{right} is the only reduction 
variable inside the loop in Figure~\ref{fig:cgc-fsm}, 
since its value is incremented by one in every loop iteration.  
When a loop finishes its execution, the value change of a 
reduction variable is a multiplication of 
the iterations executed by the loop. 

After identifying reduction variables inside a loop,
\Tool{} checks whether any exit condition of the loop is to compare 
a reduction variable with a constant number. 
If so, then the loop’s trip count is constant and \Tool{} filter out the loop. 
For example, the only reduction variable \texttt{right} is dereferenced 
in exit conditions of the loop in Figure~\ref{fig:cgc-fsm} 
and it is not compared with a constant number.
Therefore, \Tool{} does not filter out the loop and considers it 
as a potential FSM implementation 
for further analysis. 

\subsection{Pinpointing State Variables}
\label{sec:variable}
Our empirical study shows that state variables are either in integer or enumeration type
and a FSM loop conditionally conducts a state transition in each iteration. 
Therefore, a FSM loop must contain at least one memory write to an integer 
(or an enumeration) variable. 
Since transition functions need to refer to the current state, 
a value assigned to a state variable in one iteration of a FSM loop needs 
to propagate to future iterations.
Given a candidate FSM loop, 
\Tool{} leverages live variable 
analysis~\cite{live-analysis} to 
identify possible state variables, which are integer variables 
updated inside the loop and have updated values live outside the loop 
or in the next iteration. 


We illustrate this approach by taking the FSM 
in Figure~\ref{fig:cgc-fsm} for example. 
Variable \texttt{state} is an enumeration variable and it is assigned 
with a new value at 
line 17, 20, 25, 27, 29, and 35 inside the \texttt{while} loop. 
The updated values are possibly read at line 15, 23 and 32 
in the next iteration of the loop or at line 41 outside the loop.
Therefore, \Tool{} considers \texttt{state} as a possible 
state variable.  

We further eliminate false positives when identifying state variables 
by considering how a FSM conducts state transitions. 
As discussed in Section~\ref{sec:study}, 
a transition function refers to the current state to determine the next state. 
Therefore, defining the next state through writing a new value to a state variable 
is control dependent~\cite{cdg} on a predicate evaluation 
using the current value of the state variable.  
Take the FSM in Figure~\ref{fig:cgc-fsm} as an example, 
transiting to state \texttt{open\_set} at line 17
is control dependent on the evaluation of ``\texttt{state==start}''
where the value of \texttt{state} is read at line 15 and \texttt{start} is a constant. 
Transiting to \texttt{close\_double} at 
line 25 is control dependent on the 
evaluation of ``\texttt{state==open\_double}'', 
and the value of \texttt{state} is read at line 23.

\Tool{} implements this mechanism using the following two steps. 
First, for each memory write to an integer (or enumeration) 
variable inside a candidate loop, 
\Tool{} searches conditional branches inside the loop 
on which the memory write is control dependent. 
Second, \Tool{} checks whether the condition 
of a searched branch 
is data dependent on the same integer variable. 
For example, the memory write at line 17 is conducted on an enumeration variable
and it is control dependent on the underlying branch 
instruction represented by 
the \texttt{switch} at line 15 and the \texttt{case} at line 16.  
The condition of the branch is ``\texttt{state==start}'' and it is 
data dependent on the same enumeration variable \texttt{state}. 
Therefore, \Tool{} identifies \texttt{state} as a state variable. 



\subsection{Constructing FSMs}
\label{sec:tuple}
With a FSM loop and identified state variables, 
we now discuss how we construct the implemented FSM. 
Specifically, we need to figure out the five-element tuple for the identified FSM. 


\section{Experiment}

\subsection{Methodology}

\noindent\textbf{Implementation and Platform.} 
We implement \Tool{} using LLVM-7.0.0~\cite{LLVM}, 
and conduct our experiments on a Linux machine, 
with E5-2630 CPU, 32GB memory and 3.10 kernel. 

\noindent\textbf{Benchmarks.}
\Tool{} is a tool to automatically extract FSMs implemented in a program. 
Since we build \Tool{} using LLVM, 
our current implementation can only work on C/C++ programs.  
However, we believe that our algorithm is general enough 
to be extended to other programming languages. 


To evaluate \Tool{}, we collect C/C++ programs from three sources. 
First, we evaluate \Tool{} on two programs collected in a CTF contest~\cite{ctf}, 
one contains a FSM, and the other one does not. 
Second, we leverage the DARPA CGC dataset~\cite{CGC}. 
In total, there are 200 programs in the CGC dataset.
As discussed in Section~\ref{sec:study}, 
we already use 40 of them to conduct our empirical study,
so that we use the remaining 160 programs in our evaluation.
Third, we apply \Tool{} to OpenVPN~\cite{openvpn}, 
which provides an implementation of virtual private network and 
is included in software packages of every released Linux version. 

\begin{table}[h!]
\centering
\footnotesize
%\scriptsize

 %\setlength{\tabcolsep}{0.8mm}{
 \setlength{\tabcolsep}{1.6mm}{
\begin{tabular}{|l|c|c|c|c|}
\hline
\textbf{Source}   & \textbf{\# of programs} & \textbf{average LOC} & \textbf{\# of loops} & \textbf{\# of FSMs} \\ \hline \hline 
CTF     & 2              &             &    19       &   1         \\ \hline 
CGC     & 160            &             &    6607     &   61        \\ \hline
OpenVPN & 1              &             &    512      &    6        \\ \hline
\end{tabular}
}
%\vspace{0.1in}

\mycaption{tab:benchmark}
{Benchmark Information.}
{}
%  \label{tab:apps}
%\vspace{-0.1in}
\end{table}

The benchmark information is shown in Table~\ref{tab:benchmark}.
In total, we use 163 different programs to evaluate \Tool{}.
All our benchmarks are either real software or 
simplified programs from real applications. 
They are either widely-used in the real world or popular in the security community. 
They cover programs in small, medium and large sizes, 
with lines of code ranging from 0.3 thousand to more than 100 thousand.  
We believe that our benchmarks are representative 
enough to evaluate the effectiveness of \Tool{}.

\noindent\textbf{Evaluation Setting.} 
For all our benchmark programs, we manually examine all their loops and 
identify all FSM loops. 
As shown in Table~\ref{tab:benchmark}, there are in 
total 66 FSMs.
Four FSM loops contain two state variables, 
and all other FMS loops contain exact one state variables.
Therefore, there are in total 70 FSMs implemented in all our benchmarks.  
We apply \Tool{} to all benchmark programs. 
We mainly compute metrics to answer two research 
questions regarding the coverage and accuracy of \Tool{}.


\textbf{Q1. Coverage:} whether \Tool{} can identify all implemented FSMs?
 
\textbf{Q2. Accuracy:} whether \Tool{} will report loops, which are FSM loops, 
generating false positives. 


\subsection{Experimental Results}

\begin{table}[h!]
\centering
\footnotesize
%\scriptsize

 %\setlength{\tabcolsep}{0.8mm}{
 \setlength{\tabcolsep}{1.6mm}{
\begin{tabular}{|l|c|c|c|c|}
\hline
\textbf{Source}   & \textbf{\# FSM loops} & \textbf{# FSMs} & \textbf{\# FNs} & \textbf{\# FPs} \\ \hline \hline 
CTF               &   1                   &  1              & 0    & 0       \\ \hline 
CGC               &   59                  &  63             & 0    & 2        \\ \hline
OpenVPN           &   6                   &  6              & 0    & 0         \\ \hline
\end{tabular}
}
%\vspace{0.1in}

\mycaption{tab:exp}
{Experimental Results.}
{FN: false negative, and FP: false positive. }
%  \label{tab:apps}
\vspace{0.1in}
\end{table}

\noindent\textbf{Coverage.}


\noindent\textbf{Accuracy.}
\section{Applications}

\Tool{} is a tool that can automatically extract implemented FSMs in a program. 
In this section, we will discuss how the extracted FSMs can facilitate 
various network and security practices.  


\noindent\textbf{Network Verification.}

\noindent\textbf{Code Debloating.}
Code bloat refers to code in unnecessarily large size~\cite{code-bloat}.
It widely exists in production-run software~\cite{code-bloat-study}. 
If untackled, bloated code can introduce more bugs and vulnerabilities, 
lowering the security~\cite{protocol-mao}, 
and conduct resultless or redundant computation, 
degrading the performance~\cite{BloatFSE2008,XuBloatPLDI2009,XuBloatPLDI2010}. 

Many techniques have been proposed to address the code bloating problem. 
They either remove temporary object copies~\cite{BloatFSE2008,XuBloatPLDI2009,
XuBloatPLDI2010,Reusable,Cachetor} 
or eliminate functions unreached from 
\texttt{main}~\cite{container-debloating-1, 
container-debloating-2, dinghao-1}. 
None of them try to change the underlying program models.
With extracted FSMs from \Tool{}, researchers can consider 
further code debloating can be conducted through
eliminating unnecessary program logics. 
For example, given the extracted FSM in Figure~\ref{fig:cgc}, 
developers may consider removing state \texttt{open\_double} and 
state \texttt{close\_double}. 
A tool can take the FSM from \Tool{} as input and automatically 
remove code pertaining to the two states. 
The tool can test or validate whether the changed program is correct 
by launching testing, while monitoring the control flow 
inside the FSM loop and the value of the state variable \texttt{state}.

\noindent\textbf{Fuzz Testing.}
Fuzzing is an automated testing technique. Its core idea is 
to generate random inputs through randomly mutating existing inputs and 
use the random inputs to execute 
a tested program with the goal to trigger unexpected behaviors,
such as program crashes and assertion errors.
Fuzzers are evaluated by measuring code coverage. 
A better fuzzer can cover more lines of code 
or branches under a given time constraint. 

The state-of-the-art fuzzers are not good at processing 
implement FSMs.
Take the FSM in Figure~\ref{xxx} as an example. 

\section{Related Works}
\label{sec:related}
\noindent\textbf{Program analysis to extract models.} A set of work also applies various program analysis techniques 
to extract certain composing elements in programs: for example, NFactor~\cite{wu2016automatic} 
uses symbolic execution and program slicing to extract match-action tables in NF programs; 
StateAlyzr~\cite{khalid2016paving} extracts state abstractions. 
While \Tool{} extract FSMs, which is a more concise and expressive model.

\noindent\textbf{Blackbox modeling.} Another approach to get the FSM of a program is blackbox modeling. 
L* algorithm~\cite{angluin1987learning} by Angluin is the theoretical foundation, 
and it is applied in various scenarios (e.g., NF modeling~\cite{moon2019alembic}, 
protocol analysis~\cite{cho2011mace}). Compared with \Tool{}, 
the completeness and soundness of the blackbox approach are limited --- 
it is not possible to enumerate all inputs for blackbox testing (e.g., the input of infinite length).

\section{Conclusions}

\balance
{
\bibliographystyle{abbrvnat}
\bibliography{fsm} 
}
\end{document}
