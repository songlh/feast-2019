\section{Introduction}
\label{sec:intro}

A finite state machine (FSM) is a mathematical computation model 
that performs a series of predetermined actions in 
reaction to the model inputs~\cite{fsm}. 
FSMs provide a concise and expressive way to describe
program logic, so that they are widely adopted in different types of software, 
including network protocols, 
compiler and event-driven programs. 

Automatically extracting implemented FSMs from a program has 
many important applications. 
First, the implementation of an FSM may be inaccurate. By 
comparing the extracted version with its original design, one 
can detect potential mistakes residing in an implementation~\cite{protocol-bug}. 
Second, the network verification process depends on underlying FSM models 
in different components to validate the whole network's properties 
(e.g. isolation, reachability).
Right now, the FSMs fed into verification are largely handcrafted through 
manual inspection~\cite{fayaz2016buzz,SymNet}, 
which is time-consuming and error-prone. 
Third, extracted FSMs can help developers and automated program analysis tools
better understand program semantics, facilitating the building of future 
code debloating~\cite{container-debloating-1,container-debloating-2,dinghao-1} 
and fuzzing techniques~\cite{afl,Angora,youwei-1}.



Unfortunately, there has not yet been an existing algorithm that can extract 
all implemented FSMs from a program.
Existing static techniques~\cite{wu2016automatic,khalid2016paving} 
can extract certain models implemented in a program,
but their extracted models are less concise and expressive than FSM. 
Dynamic techniques~\cite{angluin1987learning,moon2019alembic,cho2011mace} 
view the whole programs as a blackbox and 
model it as one single FSM in a coarse granularity, 
failing to extract all FSMs and localize program code 
pertaining to an implemented FSM. 
Dynamic techniques highly depend on inputs used during FSM inference, 
lacking soundness and completeness.  


In this work, we present a tool \Tool{} that can effectively extract implemented FSMs 
in a program
with good coverage and accuracy. 
\Tool{} is built based on LLVM infrastructure, 
and thus it takes LLVM intermediate 
code emitted during compilation as input. 
\Tool{} utilizes static analysis techniques to search FSM implementations inside 
an input program,
and outputs a five-element tuple ($Q$, $\sum$, $\delta$, $s_0$, $F$) 
describing each identified FSM. 


We build \Tool{} in two steps. First, we conduct an empirical study 
on how FSMs are implemented in the real world. After examining 25 FSMs in the CGC 
dataset~\cite{CGC}, we find that FSM implementations rely on certain code patterns. 
For example, all of our studied FSMs are implemented in 
loops which do not take a constant trip count and a state transition operation 
is control dependent on the current state. 
Second, we design static analysis routines for the code patterns.
Our static analysis techniques can recognize suspicious FSM loops,
pinpoint variables representing FSM states, 
and synthesize the five-element tuple for each identified FSM. 
We use 160 programs from three sources to evaluate \Tool{}.
The evaluation results show that \Tool{} can identify all implemented FSMs 
with very few false positives. 




%The rest of the paper is organized as follows. 
%In Section~\ref{sec:study}, 
%we discuss our empirical study on how FSMs are implemented in real-world software.
%In Section~\ref{sec:impl}, we discuss the detailed design of \Tool{}.
%How we conduct experiments to evaluate \Tool{} is discussed in Section~\ref{sec:exp}. 
%We discuss the potential applications of extracted FSMs in Section~\ref{sec:app}
%In the last two sections, we discuss related works and conclude our paper. 



