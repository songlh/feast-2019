\section{Introduction}

A finite state machine (FSM) is a mathematical model of computation, 
which performs a series of predetermined actions in 
reaction to the model inputs~\cite{fsm}. 
FSMs provide a concise and expressive way to describe
program logic, so that they widely exist in various software programs, 
including network protocols, 
compiler and event-driven programs. 

Automatically extracting implemented FSMs in a program has 
many important applications. 
First, the implementation of a FSM may be inaccurate, 
so that comparing the extracted version with the design version 
can detect potential implementation mistakes~\cite{protocol-bug}. 
Second, the network verification process depends on underlying FSM models 
in different components to validate the whole network’s properties 
(i.e. isolation, reachability).
Right now, the FSMs fed into verification are largely handcrafted through 
manual inspection~\cite{fayaz2016buzz,SymNet}, which is time-consuming and error-prone. 
Third, extracted FSMs can help developers and automated program analysis tools
better understand program semantics, facilitating the building of future 
code debloating~\cite{container-debloating-1,container-debloating-2,dinghao-1} 
and fuzz techniques~\cite{afl,Angora,youwei-1}.


Unfortunately, there is no existing algorithm that can extract 
all implemented FSMs in a program.
Static techniques~\cite{wu2016automatic,khalid2016paving} 
can extract certain models implemented in a program,
but their models are less concise and expressive than FSM. 
Dynamic techniques~\cite{angluin1987learning,moon2019alembic,cho2011mace} 
view the whole programs as a blackbox and 
model it as one single FSM in a coarse granularity, 
failing to extract all FSMs and localize program code 
pertaining to an implemented FSM. 
Dynamic techniques highly depend on used inputs during FSM inference, 
lacking soundness and completeness.  

This paper presents a tool \Tool{} that can effectively extract FSMs in a program
with good coverage and accuracy. 
\Tool{} is a static technique. 
It takes program code as input 
and outputs a five-element tuple ($Q$, $\sum$, $\delta$, $s_0$, $F$) 
describing each identified FSM. 


We build \Tool in two steps. First, we conduct an empirical study 
on how FSMs are implemented in real world. After examining 25 FSMs in the CGC 
dataset, we find that there are clearly code patterns for FSM implementations. 
For example, all of our studied FMSs are implemented in 
loops which do not take a constant trip count, and a state transition operation 
is control dependent on the current state. 
Second, we design static analysis routines for the code patterns.
Our static analysis can recognize suspicious FSM loops,
recognize variables representing FSM states, and synthesize the five-element tuples. 
Our evaluation using programs from three sources shows that
\Tool{} can identify all implemented FSMs with very few false positives. 


The rest of the paper is organized as follows. 
In Section~\ref{sec:study}, 
we discuss our empirical study on how FSMs are implemented in real-world software.
In Section~\ref{sec:impl}, we discuss the detailed design of \Tool{}.
How we conduct experiments to evaluate \Tool{} is discussed in Section~\ref{sec:exp}. 
We discuss the potential applications of extracted FSMs in Section~\ref{sec:app}
In the last two sections, we discuss related works and conclude our paper. 



