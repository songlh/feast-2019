\section{Implementation}
\label{sec:impl}

From our empirical study in Section~\ref{sec:study},
we observe that a FSM is implemented using a loop 
whose iteration number is not constant  
and the loop conditionally updates a state variable in each iteration. 
Therefore, \Tool{} extracts implemented FSMs from a program in three steps.
First, \Tool{} looks for loops without constant iteration numbers. 
Second, \Tool{} checks whether there are state 
variable updates inside the loops to identify FSM implementations. 
Third, \Tool{} applies a suite 
of static analysis to figure out states ($Q$), 
transition functions ($\delta$), the initial state ($s_0$),
and final states ($F$) for each FSM implementation. 


\subsection{Filtering Loops with Constant Iterations}
As we discussed in Section~\ref{sec:study},
a FSM is usually implemented using a loop 
and the loop processes one input in each iteration to decide 
whether to update the FSM’s state. 
In reality, it is very rare that a FSM can arrive its final state after receiving 
a constant number of inputs. 
Therefore, loops iterating constant numbers of times in each execution 
are unlikely to be FSM implementations. 

We mainly leverage scalar evolution analysis to identify loops 
whose iteration numbers are constant. 
Scalar evolution analysis can identify reduction variables inside a loop.
Reduction variables are integer variables, whose values are updated 
in a constant strip in each loop iteration. 



\subsection{Pinpointing State Variables}


\subsection{Extracting FSMs}

