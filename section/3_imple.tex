\section{\Tool{} Design}
\label{sec:impl}

Our empirical study in Section~\ref{sec:study} 
shows that a FSM is usually implemented in a loop 
which does not take a constant trip count (or iteration number) 
and conditionally updates a state variable 
to transit to a new state in each iteration. 
Therefore, \Tool{} searches FSM loops
by first filtering loops with constant trip counts (Section~\ref{sec:constant}) 
and then identifying loops with state variable updates (Section~\ref{sec:variable}).
The ultimate goal of \Tool{} is to construct implemented FSMs in a program, 
and thus we will discuss how \Tool{} figures out the five-element tuple 
($Q$, $\sum$, $\delta$, $s_0$, $F$) 
for an identified FMS in Section~\ref{sec:tuple}. 

Algorithm~\ref{alg:fsm} shows the workflow of \Tool{}.
\Tool{} takes the source code of a program as input
and outputs the five-element tuple (line 8) 
and source code information (line 9) for each 
implemented FSM in the program.  



\begin{algorithm}[!htb]
    \caption{Finite State Machine Extraction}
    \label{alg:fsm}
    \begin{algorithmic}[1]
        \Require source code of a program: \emph{P}
        \Function {\Tool}{$P$}
        \State initialize an empty FSM set \emph{S} = \{\}
        \For{each loop $l$ in $P$}
        	\If{$l$ takes a constant trip count}
        		\State \textbf{continue}
        	\EndIf
        	 \If{$l$ has state variable updates}
        			\State t($Q$, $\sum$, $\delta$, $s_0$, $F$) $\gets$ ConstructFSM($l$)
        			\State srcInfo $\gets$ ExtractSRCInfo($l$) 
        			\State S.Insert(t, srcInfo)
        	\EndIf
        \EndFor
        \State \Return{$S$}
        \EndFunction
    \end{algorithmic}
\end{algorithm}


\subsection{Filtering Loops with Constant Trip Counts}
\label{sec:constant}

As discussed in Section~\ref{sec:study},
a FSM is usually implemented using a loop 
and the loop processes one input in each iteration to decide 
whether to transit to a new state. 
In reality, it is very rare that a FSM can arrive at a final state 
after processing a predefined, constant number of inputs.
Our empirical study confirms this intuition. 
None of our studied FMS loops take a constant trip count.  
To sum up, given a loop which iterates a constant 
number in each execution, 
the loop is unlikely to be a FSM implementation. 

We mainly leverage scalar evolution analysis~\cite{scalar-1,scalar-2,scalar-3} 
to identify loops whose trip counts are constant. 
Scalar evolution analysis can identify reduction variables inside a loop.
Reduction variables are integer variables, 
whose values are updated 
with a constant delta in each loop iteration. 
For example, variable \texttt{right} is the only reduction 
variable inside the loop in Figure~\ref{fig:cgc-fsm}, 
since its value is incremented by one in every loop iteration.  
When a loop finishes its execution, the value change of a 
reduction variable is a multiplication of 
the iterations executed by the loop. 

After identifying reduction variables inside a loop,
\Tool{} examines each exit condition of the loop and checks whether 
any of them is to compare 
a reduction variable with a constant number. 
If so, then the loop’s trip count is constant and \Tool{} filters out the loop. 
For example, only the dereferenced value of the reduction variable \texttt{right}
is used in exit conditions of the loop in Figure~\ref{fig:cgc-fsm}. 
Since \texttt{right} is the only one reduction variable inside the loop
and it is not compared with a constant number in exit conditions, 
\Tool{} does not filter out the loop and considers it 
as a potential FSM loop for further analysis. 


\subsection{Pinpointing State Variables}
\label{sec:variable}
Our empirical study shows that state variables are either in integer or enumeration type
and a FSM loop conditionally conducts a state transition in each iteration. 
Therefore, a FSM loop must contain at least one memory write to an integer 
(or an enumeration) variable. 
Since transition functions need to refer to the current state, 
a value assigned to a state variable in one iteration of a FSM loop needs 
to propagate to future iterations.
Given a candidate FSM loop, 
\Tool{} leverages live variable 
analysis~\cite{live-analysis} to 
identify possible state variables, which are integer variables 
updated inside the loop and have updated values live outside the loop 
or in the next iteration. 


We illustrate this approach by taking the FSM 
in Figure~\ref{fig:cgc-fsm} for example. 
Variable \texttt{state} is an enumeration variable and it is assigned 
with a new value at 
line 17, 20, 25, 27, 29, and 35 inside the \texttt{while} loop. 
The updated values are possibly read at line 15, 23 and 32 
in the next iteration of the loop or at line 41 outside the loop, 
so that these values are live in the next iteration and outside the loop. 
Therefore, \Tool{} considers \texttt{state} as a possible 
state variable.  


We further eliminate false positives when identifying state variables 
by considering how a FSM conducts state transitions. 
As discussed in Section~\ref{sec:study}, 
a transition function refers to the current state to determine the next state. 
Therefore, defining the next state through writing a new value to a state variable 
is control dependent~\cite{cdg} on a predicate evaluation 
using the current value of the state variable.  
Take the FSM in Figure~\ref{fig:cgc-fsm} as an example, 
transiting to state \texttt{open\_set} at line 17 by assigning \texttt{open\_set}
to \texttt{state} 
is control dependent on the evaluation of ``\texttt{state==start}''
where the current value of \texttt{state} is read at 
line 15 and \texttt{start} is a constant. 
Transiting to \texttt{close\_double} at 
line 25 is control dependent on the 
evaluation of ``\texttt{state==open\_double}'', 
and the current value of \texttt{state} is read at line 23.


\Tool{} implements this mechanism through the following two steps. 
First, for each memory write to an integer (or enumeration) 
variable inside a candidate loop, 
\Tool{} searches conditional branches inside the loop 
which the memory write is control dependent on. 
Second, \Tool{} checks whether the condition of a searched branch 
is data dependent on the value of the same integer variable. 
For example, the memory write at line 17 is conducted on an enumeration variable
and it is control dependent on the underlying conditional branch 
instruction for 
the \texttt{switch} at line 15 and the \texttt{case} at line 16.  
The condition of the branch is ``\texttt{state==start}'' and it is 
data dependent on the value of the same enumeration variable \texttt{state}. 
Therefore, \Tool{} identifies \texttt{state} as a state variable. 


\subsection{Constructing FSMs}
\label{sec:tuple}
We consider a loop as a FSM loop, if it does not take a constant trip count 
and contains updates to a state variable.
As discussed in Section~\ref{sec:study}, 
a FSM loop may contain more than one state variable. 
In this case, the loop is used to implement multiple FSMs.
With a FSM loop and identified state variables inside the loop, 
\Tool{} constructs a FSM for each state variable, 
by figuring out the five-element 
tuple ($Q$, $\sum$, $\delta$, $s_0$, $F$). 

To figure out all possible states ($Q$) of a FSM 
is equivalent to determine all possible values of its state variable. 
If a state variable is in enumeration type, 
\Tool{} recognizes all its possible values 
by examining the declaration of the enumeration type. 
For example, \Tool{} identifies all the six possible values of 
the state variable \texttt{state} (Figure~\ref{fig:cgc-fsm}) 
by inspecting the type declaration at line 1. 
If a state variable is an integer variable, \Tool{} regards 
a constant value assigned to the state variable or 
compared with the state variable as a possible state. 
The current version of \Tool{} only examines the function
containing the analyzing FSM loop, so that it may miss some states. 
Future work could inspect the whole program by 
applying an interprocedural 
value set analysis to identify more states. 


One iteration of a FSM loop processes a distinct input, 
such as a new character from a string or a new incoming package. 
For the new input, a FSM loop either refers to a different memory location 
or refers to the same location whose content is changed in each iteration.  
Take the FSM loop in Figure~\ref{fig:cgc-fsm} as an example, 
\texttt{right} is a pointer pointing to input characters.
\texttt{right} is incremented by one in each iteration at line 38, 
so that the FSM loop refers to a different memory location 
for an input to be processed in each iteration.  
After figuring out where a FSM locates its inputs, 
\Tool{} understands the type of the inputs 
and considers all possible values in that type
as the input alphabet $\sum$. For example, \Tool{} identifies 
the $\sum$ as all possible byte values 
for the FSM in Figure~\ref{fig:cgc-fsm}. 


\Tool{} mainly relies on symbolic execution~\cite{klee,s2e} 
to synthesize transition functions ($\delta$).  
Given a state variable,
\Tool{} conducts reachability analysis on CFG to search paths starting 
from an assignment site of the variable 
and ending at an assignment site. 
For each path, \Tool{} applies symbolic execution to 
collect path constraints and utilizes a constraint solver to 
detect whether there are conflicting constraints among the path constraints 
and whether the path constraints conflict with the pre-condition 
that the state variable is equal 
to the assigned value at the starting site. 
If not, \Tool{} successfully identifies a transition function, 
which transits a FSM from a state to another state pertaining to 
the values used at the starting assignment 
site and ending site respectively. 
\Tool{} figures out the input values the identified 
transition function refers to 
by analyzing the collected path constraints. 


We illustrate how \Tool{} synthesizes transition functions using 
the FSM in Figure~\ref{fig:cgc-fsm} as an example. 
Line 17 $\rightarrow$ 13 $\rightarrow$ 22 $\rightarrow$ 23 $\rightarrow$ 26 $\rightarrow$ 27
is a path from one assignment site of the state variable \texttt{state} 
to another assignment site. 
The path constraints are 
``\texttt{*right != NULL \&\& state != close\_set \&\& *right == `”’ \&\& state == open\_set}'', 
which do not contain conflicting constraints. 
The path constraints do not conflict with the pre-condition ``\texttt{state==open\_set}'' 
specified at the starting assignment site at line 17.
Therefore, \Tool{} identifies a transition function which transits the FSM from  
\texttt{open\_set} to \texttt{open\_double}. 
\Tool{} figures out the input value used by the 
transition function as `\texttt{”}', 
indicated by the path 
constraint ``\texttt{*right == `”’ }''.  
Line 17 $\rightarrow$ 13 $\rightarrow$ 14 $\rightarrow$ 15 
$\rightarrow$ 16 $\rightarrow$ 17
is another path identified by the reachability analysis. 
However, the path constraints (``\texttt{*right != NULL \&\& state != close\_set \&\& *right == `|’ \&\& state == start}’’) 
conflict 
with the pre-condition (``\texttt{state==open\_set}’’), 
and thus \Tool{} does not consider this path indicates a transition function. 



\Tool{} compute $s_0$ and $F$ through analyzing the value of 
the state value before the execution of the 
FSM loop and after the loop execution respectively. 
For example, the value of \texttt{state} is \texttt{start} before the loop in Figure~\ref{fig:cgc-fsm} executes, 
so that $s_0$ of the FSM is \texttt{start}.
The loop terminates when finishing 
parsing the input string pointed by \texttt{right},
leaving \texttt{state} to be any value declared at line 1, 
and thus $F$ consists of all states. 







