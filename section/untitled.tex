

To guide the implementation of \Tool{}, our empirical study 
is mainly conducted from the following aspects. 




Third, what is the input alphabet ($\sum$)? 
The input alphabet of a FSM is theoretically bounded by all possible values 
of the type used to save inputs. 
For example, the input alphabet of the FSM 
in Figure~\ref{fig:cgc-fsm} 
contains all possible byte values.
Sometimes, an input alphabet is a subset of all possible values, 
and we think value set analysis can help refine the input alphabet for an 
identified FSM.
We also observe that during the implementation of a FSM, 
developers usually do not enumerate the processing rule for every 
input in the alphabet,
while they tend to explicitly specify the rules only for several special inputs
and leave others to be handled by a default setting. 
For example, only the processing rules for `\verb/|/ and `''' are specified in Figure~\ref{fig:cgc-fsm}, 
and all other byte values are handled by the defaulting setting specified at line 37. 

Four, how the transition functions ($\delta$) are implemented?

Five, how to specify the initial state ($s_0$) and the final states ($F$)? 