\section{Understanding Real-World FSM Implementations}
\label{sec:study}
%In this section, we first review the mathematical definition of FSM. 
%We then describe how we conduct our empirical study 
%on real-world FSM implementations 
%and the study results. 

In this section, we first provide some background 
for the mathematical definition of FSM and 
the methodology of our empirical study. 
We then present detailed study results. 


\subsection{Background and Methodology}






%\noindent\textbf{Background.}
A finite state machine (FSM) is a mathematical computation model
that takes external inputs and transmits among a set of predefined internal states. 
At any time, an FSM can only be at one state.
When a certain condition is satisfied,
an FSM transits from one state to another.
An FSM can be specified using a five-element tuple ($Q$, $\sum$, $\delta$, $s_0$, $F$),
where $Q$ is a set of internal states, $\sum$ is an input alphabet,
$\delta$ is a set of transition functions,
$s_0$ is the initial state, and $F$ is 
a set of final states.

We inspect the DARPA CGC dataset~\cite{CGC} to
understand how FSMs are implemented in the real world.
We choose the CGC dataset because it
contains a large number of diverse programs simplified
from real-world software and it
is also widely used in security
community~\cite{QSYM, Driller, VUzzer}.

{
\begin{figure}[h]
\begin{minipage}{\columnwidth}
\begin{center}
\scriptsize
\lstinputlisting[xleftmargin=.15in,language=C++,basicstyle=\ttfamily,keepspaces=true]{figure/cgc_fsm.c}
%\caption{A data race caused by anonymous function.}
%\label{fig:docker27037}
\mycaption{fig:cgc-fsm}{A simplified FSM implementation from the CGC dataset.}
%{The code has been simplified for illustration purpose.}
{}
\end{center}
\end{minipage}
\end{figure}
}


%\noindent\textbf{Methodology.}



To conduct the study, we first randomly sample
40 programs from the CGC dataset.
We then manually inspect the sampled programs and looked for 
FSM implementations.
In total, we identify 25 implemented FSMs
and treat them as the targets of our study.
Figure~\ref{fig:cgc-fsm} shows one such example.
Function \texttt{cgc\_parse\_set()} takes string \texttt{right}
as input and returns \texttt{true} if \texttt{right} matches
regular expression ``\verb/|("[^"]*")?|/''.
Figure~\ref{fig:cgc} shows the underlying FSM.
In total, the implemented FSM contains six different states
and nine possible state transitions.



\begin{figure}[htb]
\centering
\includegraphics[width=2.5in]{figure/cgc-fsm.pdf}
%\vspace{-1.5em}
\mycaption{fig:cgc}{An implemented FSM in the CGC dataset.}
%{The code has been simplified for illustration purpose.}
{}
\end{figure}

%\noindent\textbf{Empirical Study.}

\subsection{Study Results}
To guide the design of \Tool{}, our empirical study
is mainly conducted to answer the following five questions. 



\noindent{{\textit{\textbf{Q1.} what code constructs are used to implement the FSMs?}}}
Since our goal is to statically identify and extract implemented FSMs,
we must know what code constructs to inspect.
Not surprisingly, all our studied FSMs are implemented using a loop,
like the \texttt{while} loop at line 13 in Figure~\ref{fig:cgc-fsm}.
In each loop iteration, an implemented FSM processes an input and
determines whether to stay in the current state or transit to a new state.
The underlying intuition is that an FSM usually needs to process
multiple inputs and similar logics are applied during the processing,
so that using a loop is a natural way to implement an FSM.

Another important observation is that
an FSM loop usually does not execute constant
iterations or take a constant trip count.
Its execution dynamically depends on its inputs. 
Intuitively, it is very rare that an FSM can arrive at a final
state after processing a predefined, constant number of inputs.
Take the FSM implemented in Figure~\ref{fig:cgc-fsm} as an example,
the iteration number of the \texttt{while} loop is not constant and
when the loop terminates its execution  
depends on the content of string \texttt{right}.

\noindent{\textit{\textbf{Q2.} how internal states ($Q$) are maintained by the FSMs?}}
Intuitively, there must be a state variable to track the current state of an FSM.
The value of the state variable changes when a state transition happens.
Our study confirms this intuition.
We also find that state variables are either in integer type or in enumeration type,
and their values are discrete and bounded in a certain range.
This finding indicates that static value set analysis~\cite{DEEPVSA,VSA}
can potentially determine all possible states for an implemented FSM.
For example, the local variable \texttt{state} declared at line 11
is the state variable in Figure~\ref{fig:cgc-fsm}.
It is in enumeration type.
In total, it has six possible values
specified during its type declaration at line 1,
corresponding to the six states in Figure~\ref{fig:cgc}.
Interestingly, one studied FSM loop contains two state variables,
which reminds us that developers could use one loop
to implement multiple FSMs.
We need to extract all of them when designing \Tool{}.



\noindent{\textit{\textbf{Q3.} what are the input alphabets ($\sum$)?}}
The input alphabet of an FSM is theoretically bounded by 
the data type used to represent inputs. 
For example, the input alphabet of the FSM
in Figure~\ref{fig:cgc-fsm} contains all possible byte values.
There are also cases where an input alphabet is only a subset 
of all possible values in a particular type, 
and thus we think value set analysis should be applied to help refine
the input alphabet of an identified FSM.


We have two observations about how an FSM loop handles inputs. 
First, an FSM loop processes a distinct input in each iteration.
Sometimes, an FSM loop needs to refer to a different 
memory location for a new input.
Sometimes, a new input is written to the same
location before an FSM loop 
starts its procession in an iteration.
For example, \texttt{right} in Figure~\ref{fig:cgc-fsm} 
points to the input character
processed by the FSM loop in each iteration.
The value of \texttt{right} is incremented by one at line 38,
so that the FSM loop reads an input from a different memory 
location in each iteration.
Second, when implementing an FSM,
developers usually do not explicitly implement the 
processing rule for every possible input value. 
Instead,
they tend to specify rules only for several important input values and 
leave the others to be handled by a default rule.
For example, only the processing rules for \texttt{`|'} and \texttt{`"'}
are explicitly specified in Figure~\ref{fig:cgc-fsm},
and all the other byte values are handled by
the default rule at line 31.


\noindent{\textit{\textbf{Q4.} how the transition functions ($\delta$) are implemented?}}
One transition function is executed in each iteration of an FSM loop. 
It produces the next state for the FSM to transmit to based on 
the current state and the current input value. 
We observe that transition functions are implemented
using control-flow constructs (e.g., \texttt{if}, \texttt{switch}).
For example, a transition function in Figure~\ref{fig:cgc-fsm}
is implemented at line 14, 15, 16, and 17.
If the current state is \texttt{start}
at line 16 and the current input value is
\texttt{`|'} at line 14, the transition function outputs \texttt{open\_set}
as the next state at line 17.
Line 22, 23, 24, and 25 implement another transition function,
which consumes an input character \texttt{`"'} at line 22 and
transits from the current state
\texttt{open\_double} at line 24 to
the next state \texttt{close\_double} at line 25.



\noindent{\textit{\textbf{Q5.} how to specify the initial states ($s_0$) and the final states ($F$)?}}
$s_0$ of an FSM can be specified
by the value of the state variable before the execution of the corresponding 
FSM loop.
For example, the initial state $s_0$ of the FSM in Figure~\ref{fig:cgc-fsm}
is \texttt{start}, which is the value of the variable \texttt{state}
before the loop execution at line 13.
When an FSM loop finishes its execution,
all possible values of its state variable
constitute $F$ of the FSM.
Take the FSM implemented in Figure~\ref{fig:cgc-fsm} as an illustration, 
the \texttt{while} loop terminates its execution when 
finishing parsing the input string \texttt{right}. 
At that time, \texttt{state} can be any of the six values specified
in the type declaration at line 1. 
Therefore, any state of the FSM in
Figure~\ref{fig:cgc} can be a final state.


\noindent\underline{{\textit{Discussion.}}}
Our empirical study shows that FSM implementations follow certain 
code patterns. For example, all our studied FSMs are implemented using a loop, 
and state transitions are implemented in a way that 
the next value of a state variable is control dependent 
on the current value of the state variable 
and the current input. 
This finding motivates us to leverage static analysis to 
identify FSM implementations 
through matching the code patterns. 
We will present more details later in Section~\ref{sec:impl}.

Threats to the validity of our study could come from several aspects. 
All our studied FSM implementations come from the CGC dataset,
which is designed for security purposes and only contains simplified programs 
instead of real software. 
Although all the studied FSMs are implemented in a similar way, 
we do believe that there are other methods to implement an FSM, 
such as using an event handler or a recursive function. 
Despite these limitations, all our findings are summarized 
after inspecting a relatively large number of randomly sampled FSM implementations. 
We believe our findings are intuitive and general enough to represent a high percentage 
of FSMs in real-world software. 



%To sum up, our empirical study shows that
%there are certain code patterns leveraged by developers to implement FSMs.
%In Section~\ref{sec:impl}, we will present how we
%leverage these patterns to build \Tool{},
%which can automatically extract implemented FSMs from a program.






