\section{Empirical Study}
In this section, we will first review the mathematical definition of FSM and then 
we will describe our empirical study on how FSMs are implemented 
in real-world software. 

\noindent\textbf{Background.}
A finite state machine (FSM) is a mathematical computation model, 
which consists of several internal states and takes inputs from the outside.
At any time, a FSM can only be in one state. 
When a certain condition is satisfied, 
a FSM transits from one state to another state. 
A FSM can be specified using a five-element tuple ($Q$, $\sum$, $\delta$, $s_0$, $F$),
where $Q$ is a set of internal states, $\sum$ is an input alphabet, 
$\delta$ is a set of transition functions,
$s_0$ is an initial state, and $F$ is a set of final states. 

\noindent\textbf{Real-World Implementation.}
We leverage the DARPA CGC dataset~\cite{CGC} to 
understand how FSMs are implemented in the real world. 
We choose the CGC dataset, because it 
contains diverse programs simplified 
from real-world software and it is also widely used in security 
community~\cite{QSYM, Driller, VUzzer}. 


To conduct the study, we first randomly sample 
40 programs from the CGC dataset.
We then manually inspect the sampled programs and look for implementation of FSMs.
In total, we identify {\color{red}{XXX}} implemented FSMs, 
and they are the targets of our study.
