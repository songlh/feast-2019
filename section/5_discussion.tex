
\section{Applications}
\label{sec:app}

%\Tool{} is a tool that can automatically extract implemented FSMs in a program.
In this section, we will discuss how the extracted FSMs can facilitate
various network and security practices.


\noindent\textbf{Network Verification.}  In network operation, before a network
is deployed into production, its configuration needs to be verified to avoid
runtime errors. In such a network verification process, network operators
usually build behavior models for individual network appliances and then
reason about the end-to-end properties of the
network~\cite{mai2011debugging,khurshid2013veriflow,kazemian2012header,kazemian2013real,fayaz2016buzz,panda2017verifying}.
FSM is an expressive
behavior model to represent a wide range of network appliances, including
switches and software network functions (e.g. load balancers, firewalls, NAT). 
With individual FSMs and the network topology ready, the network
operator could verify whether the communication between end hosts satisfies
properties (e.g. reachability, isolation, loop-freedom).%~\cite{xxx}.

\Tool{} is helpful in the procedure of ``building behavior models for
individual network appliances'', which currently is manually crafted by
the network operators by reading the code or according to their
empirical understanding. \Tool{} can primarily automate the
transformation from network software to FSMs. More importantly, it
provides the confidence that the output FSM is logically equivalent
to the original software.


\noindent\textbf{Code Debloating.}
Code bloat refers to codes in an unnecessarily large size~\cite{code-bloat}.
It widely exists in production-run software~\cite{code-bloat-study}.
If untackled, bloated codes can introduce vulnerabilities~\cite{protocol-mao} and 
degrade the software performance~\cite{BloatFSE2008,XuBloatPLDI2009,XuBloatPLDI2010}.
Many techniques are proposed to address the code bloating problem.
They either remove temporary object copies~\cite{BloatFSE2008,XuBloatPLDI2009,
XuBloatPLDI2010,Reusable,Cachetor}
or eliminate functions unreached from
\texttt{main}~\cite{container-debloating-1,
container-debloating-2, dinghao-1}.
None of them change the underlying program models.
With extracted FSMs from \Tool{},
further code debloating can be performed through
eliminating unnecessary program logics.
For example, given the extracted FSM in Figure~\ref{fig:cgc},
developers may consider removing state \texttt{open\_double} and
\texttt{close\_double}.
A tool can take the FSM as input and automatically
remove code pertaining to the two states.
The tool can test or validate the changed program
by monitoring the control flow
in the FSM loop and the value of the state variable.

\noindent\textbf{Fuzz Testing.}
Fuzzing is an automated testing technique,
which executes a program
using randomly mutated inputs
with the goal to trigger unexpected program behaviors,
such as crashes and assertion errors~\cite{afl,Angora,youwei-1}.
Fuzzers are usually evaluated by measuring code coverage.
A better fuzzer can cover more lines of code
or branches under a given time constraint.
The state-of-the-art fuzzers are not good at processing
FSMs in a program.
Take the FSM in Figure~\ref{fig:cgc-fsm} as an example,
``\verb/||/'' is the only input which has two characters and
can transit the program to state \texttt{close\_set} at line 20.
If a fuzzer completely relies on random mutation, the probability to
generate ``\verb/||/'' is very low ($1/(256 * 256)$).
However, if the fuzzer is enhanced by the
FSM in Figure~\ref{fig:cgc},
it will easily figure out how to create inputs to quickly cover
all states and state transitions.






